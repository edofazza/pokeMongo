\section{Introduction}
\textbf{\textit{PokeMongo}} is a gaming application in which \textbf{Users} compete each other to build up the best \textbf{Team} choosing between the set of \textbf{Pokémons} available. 

\subsection{Description}
Every \textbf{User} can build up his own \textbf{Team}. Every \textbf{Team} is composed by up to 6 distinct \textbf{Pokémon} and is assigned to a numerical value (points) based on features and properties of the chosen \textbf{Pokémon}, for ranking purposes.\medskip \\
A \textbf{User} can also follow other \textbf{Users} in order to make new friends basing on common friends or common interests. Moreover \textbf{Users} can express sentiments on \textbf{Pokémon}, choosing their favourite ones and posting or commenting on them. \medskip \\
\textbf{Users} can also navigate through the ranking in order to visualize the best \textbf{Teams} (according to the values cited before) and the most used/caught \textbf{Pokémon}, both among their friends, grouped by \textit{country} and among worldwide players.\medskip \\
\textbf{User} can browse for a specific \textbf{Pokémon} using the \textit{Pokédex} tool, in which he/she can lookup for \textbf{Pokémon} according to search filters like \textit{Pokémon name}, \textit{Type} or \textit{Point}s.\medskip \\
Moreover, as a “real” Pokémon Trainer, the \textbf{User} is invited to \textit{Catch ‘em all}, i.e. to try to get a new \textbf{Pokémon} in order to create/update his/her own \textbf{Team}. Thus, it is provided to the \textbf{User} a prefix number of \textit{daily Pokéball} to be used to try to capture them. At each \textbf{Pokémon} is associated a probability to catch it, the higher the \textbf{Pokémon}’s value, the lower the probability.\medskip \\
Furthermore, the \textbf{User} can exploit the social network structure of the application to make new friends and discover new \textbf{Pokémon}. Indeed, he/she can search for new friends by \textit{username} or choosing them among the provided \textit{recommended friends list}. 
The \textbf{User} can choose his/her \textbf{favourite Pokémon}, obtaining in this way a shortcut to catch it faster, and can post or answer to \textbf{Posts} in order to express his/her opinion on that \textbf{Pokémon}. \medskip \\
In addition, to extend the dynamic behaviour of the application, the \textit{catch rate} (i.e. the probability to get a Pokémon using a Pokéball) changes in time depending on the number of \textbf{Users} who have that \textbf{Pokémon}: \textit{the more it is popular, the harder will be to catch it}. Since the rankings’ points are computed based on the \textit{catch rate}, the winning strategy could be on predicting which \textbf{Pokémon} will become popular in the near future and try to get it early! Every \textbf{User} has access to the visualization of the temporal drift of the \textit{catch rate}. \medskip \\
The safeguard and the improvement of the application is in charge of \textbf{Admin} users. They are able to \textit{ban mischievous users}, \textit{delete inappropriate posts or comments}, \textit{add/remove Pokémon} to the collection, \textit{consult geo-temporal usage statistics} which are useful to make new business plans. \medskip \\
